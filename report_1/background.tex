\section{Background}
In this section some basic concept, necessary in the following, are introduced.
\par
The walking task can be easy summarize using the \emph{walking cycle} (Figure (%\ref{fig:walking_graph}))
In the following \emph{Step} is defined as a half of walking cycle, which can be divided into two
phases, \emph{Double support} and \emph{Single support}.
In the first, only one foot is in contact with the ground; while in the other one boot feet are
in contact with the ground.
The Single support phase is made up by three sub-phase:
\begin{itemize}
\item[-] \emph{Detachment}:
\item[-] \emph{Swing phase}:
\item[-] \emph{Impact}:
\end{itemize}

\subsection{Swing phase}
In this subsection the Swing phase is analyzed and modelled using the 3D Linear Inverted
Pedulum Model (3D-LIPM).
\par
From the viewpoint of control and walking pattern generation the swing phase can be modelled using
two different approach. In the first one the precise knowledge of robot dynamics
(mass of each joint, location of center of mass of each joint, etc.) is required.
The second approach uses only a limited knowledge of the robot dynamics (the position of whole body center of mass ($CoM$), the entire mass of the robot ($M$), etc.). This last approach, called
\emph{the inverted pendulum approach}, is used in the following report to modelling the swing phase.
\newpage
